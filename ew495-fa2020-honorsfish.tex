\documentclass[twocolumn,10pt]{IEEEtran}

\newcommand{\myroot}{.}
\usepackage{wrcecapstone}

\title{Bio-inspired flapping foil propulsion for enhanced maneuverability of unmanned underwater vehicles}
\author{MIDN 1/C J J Kenneally, III and MIDN 1/C E L  Klatt\thanks{Authors are with the Department of Weapons, Robotics, and Control Engineering at the United States Naval Academy. Addresses for correspondence \emph{m213402@usna.edu} and \emph{m213546@usna.edu}.}}
\date{December 3, 2020}

% For EW495 title page
\usepackage{titlepage4956}
\coursenumber{EW495}
\student{MIDN 1/C J J Kenneally, III and MIDN 1/C E L Klatt}
\advisor{Assistant Professor D. Evangelista}
%\coverpicture{\includegraphics[height=1.88in]{\myroot/figures/problem-statement-1a.png}}

% for IEEE style citations
\bibliographystyle{IEEEtran}
\usepackage[noadjust]{cite}

\begin{document}
%\maketitlepage
\maketitle
\begin{abstract}
High $L/D$ unmanned underwater vehicles (UUVs) are typically optimized for ahead propulsion and do not maneuver well in confined environments or during launch and recovery.  We are developing \textbf{``pop-out'' bioinspired flapping foil auxiliary propulsion} for UUVs based on \textbf{fish locomotion} in the \emph{Tetraodontiformes}, including pufferfish, triggerfish, and \emph{Mola mola}, the ocean sunfish.  This group uses median/paired fins (MPF), other than the caudal fin, to accomplish maneuvers in forward and lateral directions seamlessly.  During fall 2020, our efforts have developed an idealized 2D model of flapping, to guide early design, as well as initial hardware for hydrodynamic testing. In spring 2021, we plan to demonstrate bio-inspired flapping fins on a high $L/D$, cylindrical midbody shape, to test potential improvements in low speed maneuverability.
\end{abstract}

\begin{IEEEkeywords}
keywords here
\end{IEEEkeywords}

% for stuff from 502 you want me to add in just tell me what sections/pages from your 502 reports and I'll get them in here and then smooth them over

\section{Introduction}
Fish are maneuverable; UUVs not as much. 
\begin{figure}
\begin{center}
\includegraphics[width=\columnwidth]{figures/fig1.png}
\end{center}
\caption{\textbf{Fish have many fins; some are used for auxiliary propulsion and maneuvering.}  Fish in the \emph{Tetraodontiformes} use median and paired fins (MPF), other than the caudal fin, to maneuver.}
\end{figure}
\begin{figure}
\begin{center}
\includegraphics[width=\columnwidth]{figures/fig2.png}
\end{center}
\caption{Large $L/D$ torpedo-shaped UUVs often do not maneuver well. \textbf{Can we apply fish fins to UUVs for naval applications that require maneuvering in confined areas, recovery and handling, etc.}}
\end{figure}

\subsection{Toy model}
\begin{figure}
\begin{center}
\includegraphics[width=\columnwidth]{figures/fig4.eps}
\end{center}
\caption{Example control map from 2D analysis in Matlab. Map will eventually be used for high level motion planning.}
\end{figure}
\begin{itemize}
\item Assume low advance ratio $J$
\item Obtain relative velocity and $\alpha$
\item Use empirical $C_L$, $C_D$
\item Rotate resultant forces to vehicle coordinates
\item Create a map of force for foil sway $v$, angle $\psi$
\end{itemize}


\subsection{3-axis test rig}
The tethered test rig will be implemented next semester.  Utilizing the SURF in Hopper Hall, hardware experimentation will occur.  The goal of the test rig is to create a foil pattern for desired motion of the unmanned underwater vehicle.  Various phase angles will be tested to produce the desired angular movement of the UUV for low speed maneuverability.  Additionally, the results from experimentation will result in the compilation of a blending algorithm for seamless operation at low speed.  The algorithm will incorporate horizontal and vertical movement to ensure 3-axis functionality.  The ultimate goal is for the blending algorithm and the phase angle pattern is to enable low speed maneuverability of the UUV.  Low speed maneuverability enables recovery of UUVs and improved functionality in confined environments.  

\subsection{Modeling of the base UUV}
% If you want to use a figure provide me with .png or .jpg or whatever and I can put it in. For a table put it in comments and I can format for you. _____ please attach some sketches from the folder, no need for a table
The elementary version of the UUV will be a 4 inch diameter PVC pipe.  The pipe will be capped at the ends.  Utilizing blue foam, the fundamental UUV will be neutrally buoyant in water.  Neutral buoyancy will also need to incorporate the density of the servo motors and the Tetrix pieces used.  3D printed fish fins will attach to the UUV body by pieces of a Tetrix kit.  The Tetrix pieces will be attached to a servo motor to enable the fins to reach a desired phase angle for experimentation.  For the recreated fish fins, the current intention is to utilize resin printing to allow for flexibility in the fins to better mimic that found in fish.  The fin structure will be that of an ocean sunfish ("mola mola") or a triggerfish. 
 
\subsection{Actuator designs}
% Do you want me to use your sketches from your poster directory?
include sketches from poster please
Servo actuators will attach to the body of the UUV.  The actuators are connected to the fins by Tetrix pieces.  The fins will passively pop out and adjust its angular position to the desired phase angle sent to the servo.  Servos will receive desired angles from the mbed or raspberry pi to move the position of the fins to the phase angle.  

\section{Methods and materials}
\subsection{Toy model}
Evangelista will copy from poster
\subsection{3-axis test rig}
The test rig will have a tethered power source.  This power source will also eliminate the need to hardcode the mbed or raspberry pi.  A tether power source will simplify data collection and streamline the experimentation process in order to complete more testing to obtain a larger pool of results.  Waterproof servos will attach to the UUV body.  Communicating using the raspberry pi or mbed, the servo will move the fish fin to the desired phase angle and the ensuing motion will be recorded via accelerometers and video footage.  The video will be analyzed in Matlab via the DLTviewer application.  To simplify testing, the rig will focus on a single plane.  Once mastery is achieved over a single plane, another plane will become the focus.  As experimentation of each plane is completed, results will be blended to create the algoithm motion.  Completed results will be analyzed to to identify the foil pattern for desired motion of the UUV. 
%% please incorporate one of the images of the 3d printed fin
\section{Results}
\subsection{Toy model}
Evangelista will copy from poster
\subsection{3-axis test rig}
There are no current results from the 3-axis test rig.  The intention is to get into the Hopper Hall SURF early in the spring to begin test rig.  Results for the test rig come in the form of 3D printed fins.  Due to COVID-19 restrictions, limited progress could be made in 3D printing fins.  Additionally, there was a need to relearn using Autodesk and Makerbot.  An elementary fin was printed, but details of dimensions need refined.  Moreover, the plastic material common for 3D printing creates a rigid structure inconsistent with the flexible material of fish fins.  To alleviate this inconsistency, the intention is to utilize resin material to print appropriately dimensioned fish fins mirroring the pectoral fins of the mola mola or triggerfish.

\begin{figure}
\begin{center}
\includegraphics[height=\columnwidth]{figures/fig6.png}
\end{center}
\caption{Exploded view of initial prototype of bioinspired fin for hydrodynamics testing. First axis provides lateral movement (sway); second axis provides rotation (yaw).}
\end{figure}

\begin{itemize}
\item Simplified model requires empirical testing
\item 2 or 3 axes, modular fin
\item Hitec HS646WP actuators
\item Arduino or Raspberry Pi
\item ROS2 interface, mockup of final control scheme
\item Potential \textbf{instrumented smart fin}, with proprioception, force \& flow sensing
\item Fin flexibility, shape
\end{itemize}

\subsection{Modular fin designs}
\begin{figure}
\begin{center}
\includegraphics[width=\columnwidth]{figures/fig7.png}
\end{center}
\caption{Solid model of example fin design and resulting 3D print. \textbf{Modular design} allows swap out of alternative fin shapes, stiffnesses, or \textbf{smart fins} with embedded instrumentation.}
\end{figure}

\section{Discussion}
\subsection{What We Learned}
low speed definition

Neutral buoyancy
Refamiliarize servos
3D printing
waterproofing hardware
advance ratio
bollard condition
flexure
representing fin in simulink
vehicle dynamics


\subsection{Challenges Going Forward}
what we're trying to do

Looking ahead to next semester, the goal is to apply fish fins to hardware UUVs.  

Power supply in underwater environment

effective implementation of fins to produce maneuverability data
lab time limitations
Sensing and compiling data

Simulation challenges
 thrust forces
 drag coeffecients and considerations
 control algorithm connecting fins to directionality
 raspberry pi
 
\begin{itemize}
\item Actuator testing
\item Testing on free, cylindrical body
\item \textbf{Use of Hopper Hall SURF} 
\item Synergy with turtle fin \textbf{amphibious} effort
\item Initial \textbf{instrumented smart fin}
\end{itemize}

\subsection{Control of fin stroke}
If we consider a complete fin stroke as a trajectory through the control map, \textbf{can we use variational methods to find an ``optimal'' stroke?}
\begin{itemize}
\item \textbf{Highest force} production in a desired direction
\item \textbf{Most efficient} in a desired direction
\item \textbf{Least movement} for a required force
\end{itemize}

\section*{Acknowledgements}
We thank our sponsor W Sandberg. We thank A Laun and D Fredriksson for taking interest in our project and facilitating the simulation creation.  We thank CAPT B Baker for assistance with rapid prototyping. Finally, we thank our sister project to develop turtle flippers, MIDN 1/C C Ashley, G Brothers, and C Douglas; and MIDN 2/C C Smith. The project is also funded by Lockheed Martin.

% Don't worry about references at all I can handle that
\bibliography{IEEEabrv,\myroot/references/honorsfish.bib}

\begin{IEEEbiography}%
[{\includegraphics[width=1in,height=1.25in,clip,keepaspectratio]{\myroot/figures/M213402.jpg}}]%
{James J Kenneally III} is a midshipman at the United States Naval Academy majoring in Robotics and Control Engineering (Honors). Upon graduation, he will do (service selection here). 
\end{IEEEbiography}

\begin{IEEEbiography}%
[{\includegraphics[width=1in,height=1.25in,clip,keepaspectratio]{\myroot/figures/M213546.jpg}}]%
{Evan L Klatt} is a midshipman at the United States Naval Academy majoring in Robotics and Control Engineering (Honors). Upon graduation, he will do (service selection here). 
\end{IEEEbiography}
\end{document}