\documentclass[twocolumn,10pt]{IEEEtran}

\newcommand{\myroot}{.}
\usepackage{wrcecapstone}
\usepackage{balance}

\hypersetup{%
	pdfauthor={J J Kenneally, III and E L Klatt},
	pdftitle={Bio-inspired flapping foil propulsion for enhanced maneuverability of unmanned underwater vehicles},
	pdfsubject={bioinspired design},
	pdfkeywords={fish, fins, UUV, bioinspiration, biomimetics, maneuvering, agility}}

\title{Bio-inspired flapping foil propulsion for enhanced maneuverability of unmanned underwater vehicles}
\author{MIDN 1/C J J Kenneally, III and MIDN 1/C E L  Klatt\thanks{Authors are with the Department of Weapons, Robotics, and Control Engineering at the United States Naval Academy. Addresses for correspondence \emph{m213402@usna.edu} and \emph{m213546@usna.edu}.}}
\date{December 3, 2020}

% For EW495 title page
%\usepackage{titlepage4956}
\usepackage{wrcetitlepage}
\coursenumber{EW495}
\student{MIDN 1/C J J Kenneally, III and MIDN 1/C E L Klatt}
\advisor{Assistant Professor D. Evangelista}
%\coverpicture{\includegraphics[height=1.88in]{\myroot/figures/problem-statement-1a.png}}

% for IEEE style citations
\bibliographystyle{IEEEtran}
\usepackage[noadjust]{cite}

\begin{document}
\maketitlepage
\maketitle
\begin{abstract}
High $L/D$ unmanned underwater vehicles (UUVs) are typically optimized for ahead propulsion and do not maneuver well in confined environments or during launch and recovery.  We are developing \textbf{``pop-out'' bioinspired flapping foil auxiliary propulsion} for UUVs based on \textbf{fish locomotion} in the \emph{Tetraodontiformes}, including pufferfish, triggerfish, and \emph{Mola mola}, the ocean sunfish.  This group uses median/paired fins (MPF), other than the caudal fin, to accomplish maneuvers in forward and lateral directions seamlessly.  During fall 2020, our efforts have developed an idealized 2D model of flapping, to guide early design, as well as initial hardware for hydrodynamic testing. In spring 2021, we plan to demonstrate bio-inspired flapping fins on a high $L/D$, cylindrical midbody shape, to test potential improvements in low speed maneuverability.
\end{abstract}

\begin{IEEEkeywords}
fish, fins, UUV, bioinspiration, biomimetics, maneuvering, agility
\end{IEEEkeywords}

% for stuff from 502 you want me to add in just tell me what sections/pages from your 502 reports and I'll get them in here and then smooth them over

\section{Introduction}
Fish are maneuverable; UUVs not as much. Flapping foil propulsion is widely distributed in animals. In flapping foil propulsion, a foil is moved through the fluid (water in this case) to generate forces. The simplest form of flapping in 2D takes the case of in plane motions (e.g. surge and sway plus yaw; or surge and heave plus pitch). To achieve efficient and approximately unidirectional force generation, motions are generally periodic, non-symmetric, and require a proper phase relationship between rotation and translation of the foil in order to achieve useful angles-of-attack.
\begin{figure}[h]
\begin{center}
\includegraphics[width=\columnwidth]{figures/fig1.png}
\end{center}
\caption{Fish have many fins; some are used for auxiliary propulsion and maneuvering.  Fish in the \emph{Tetraodontiformes} use median and paired fins (MPF), other than the caudal fin, to maneuver.}
\end{figure}
\begin{figure}[h]
\begin{center}
\includegraphics[width=\columnwidth]{figures/fig2.png}
\end{center}
\caption{Large $L/D$ torpedo-shaped UUVs often do not maneuver well. Can we apply fish fins to UUVs for naval applications that require maneuvering in confined areas, recovery and handling, etc.?}
\end{figure}

\subsection{Force generation by a foil}
The forces (and torque) generated by a foil take the following form:
\begin{align}
\mbox{lift} &= C_L 0.5\rho U^2 A \\
\mbox{drag} &= C_D 0.5\rho U^2 A \\
\mbox{moment} &= C_M 0.5\rho U^2 A \lambda
\end{align}
where $C_L$, $C_D$, and $C_M$ are the lift, drag, and moment coefficients, respectively; $\rho=\SI{1.030}{\kilo\gram\per\meter\cubed}$ is the density of seawater, $U$ is the speed of the foil through the fluid, $A$ is a characteristic area and $\lambda$ is a characteristic length. Lift is perpendicular to $U$, and drag is anti-parallel to $U$.

The nondimenstional coefficients ($C_L$, $C_D$, and $C_M$) relate forces to the flow ($\rho$ and $U$) and size/geometry ($A$, $\lambda$). In general, these are functions of various important parameters such as angle of attack ($\alpha$), Reynolds number ($\operatorname{Re}=\frac{UL}{\nu}$), Froude number ($\operatorname{Fr}=\frac{U}{\sqrt{gL}}$), Strouhal number $\operatorname{St}=\frac{fL}{U}$, etc. Here we consider primarily the dependence of these coefficients on angle of attack ($\alpha$), or the angle at which the foil slices through the fluid. The coefficients themselves are determined experimentally through wind tunnel or hydrodynamic testing and may appear as in \fref{fig:intro-coeffs}. 
\begin{figure}
\begin{center}
\includegraphics[width=\columnwidth]{\myroot/figures/fig3-coeffs.eps}
\end{center}
\caption{Lift, drag, and moment coefficients for a flat plate, after \cite{moore2014, cory2008, tangler2005}.}
\label{fig:intro-coeffs}
\end{figure}

\Fref{fig:intro-coeffs} shows the lift, drag, and moment coefficients for flat plate. The lift coefficient ($C_L$) increases with $\alpha$ to a point, after which the foil stalls, reducing the lift generated. In contrast, the drag coefficient ($C_D$) is lowest at $\alpha=0$ and increases as the foil is turned broadside to the flow, approaching values for a flat plate at $\alpha=\ang{90}$. 

The lift and drag act at an effective center of pressure and can equivalently be thought of as creating a moment about a fixed station. Changes in moment coefficient ($C_M$) reflect the chordwise movement of the center of pressure ($C_P$) as the $\alpha$ changes; these become important in actuating the foil, since small moment around the quarter-chord point minimizes the necessary actuating torque. Moments are also of critical importance in underactuated flapping foil designs in which the foil is compliantly mounted and must passively rotate to a useful $\alpha$.











\section{Methods and materials}
\subsection{Simplified 2D model}
A fully 3D simulation of a flapping foil with unsteady effects is beyond the scope of undergraduate projects, but a simplified simulation of a 2D foil may provide insight. The simplifying assumptions are:
\begin{enumerate}
\item Vehicle speed is approximately constant during a flapping cycle. This is valid when the flapping foil actuator forces generated are small compared to the inertia of the UUV. 
\item Foil performance is determined by the (static) coefficients $C_L$, $C_D$, and $C_M$ and their dependence on $\alpha$, $\rho$, $U$ and $A$ only. We ignore unsteady effects, as well as Reynolds number dependence. 
\item To consider realistic servo actuators, the foil moves with prescribed trajectory in $y$ and $\theta$, but is limited in the lateral extent of travel $Y_{max}$, maximum rotation $\theta_{max}$, and maximum speeds of movement. 
\end{enumerate}

\subsection{Simulation}
Consider a coordinate system aligned with the vehicle. We define $(U,V)$ as the (assumed steady) speed of the vehicle through fluid. We also define $v$ as the lateral speed of the flapping foil actuator relative to the vehicle. Thus, the relative velocity of fluid with respect to the flapping foil actuator is:
\begin{equation}
\vec{u}_{rel} = (U, V+v).
\end{equation}

The angle of attack of the foil is given by the difference of the foil angle ($\theta$) and the angle of the flow ($\theta_f$) (both with respect to a coordinate system aligned with the body), or
\begin{equation}
\alpha = \theta - \operatorname{atan2}(U,V+v)
\end{equation}
where a four-quadrant $\arctan$ is used to ensure the correct quadrant. Thus, the forces and moments generated are
\begin{align}
\mbox{lift}\ L &= C_L(\alpha) 0.5\rho(U^2+(V+v)^2) A \\
\mbox{drag}\ D &= C_D(\alpha) 0.5\rho(U^2+(V+v)^2) A \\
\mbox{moment}\ M &= C_M(\alpha) 0.5\rho(U^2+(V+v)^2) A\lambda, 
\end{align}
however, the forces are aligned with respect to the angle of the flow, $\theta_f=\operatorname{atan2}(U,V+v)$. Rotating back to coordinates aligned with the vehicle, 
\begin{align}
\mbox{ahead force}\ X &= L\sin\theta_f + D\cos\theta_f \\
\mbox{side force}\ Y &= L\cos\theta_f - D\sin\theta_f.
\end{align}

Simulation of the actuator proceeds according in \Matlab\ to the following pseudocode:
\begin{itemize}
\item Assume low advance ratio $J$
\item Obtain relative velocity and $\alpha$
\item Use empirical $C_L$, $C_D$
\item Rotate resultant forces to vehicle coordinates
\item Create a map of force for foil sway $v$, angle $\psi$
\end{itemize}

\subsection{Control map}
For control purposes, it is useful to generate a map that gives the resultant force $(X,Y)$ as a function of $(U,V,\theta,v)$. A simple online control scheme might be to use table lookup or simplified basis functions to quickly obtain a motion that will provide force in a desired direction. 






\subsection{3-axis test rig}
Because the simple model is limited to 2D, a fully 3D 3-axis test rig is needed to provide a way to experimentally verify fin kinematic patterns and force estimates. The tethered test rig will be implemented next semester.  Utilizing the SURF in Hopper Hall, hardware experimentation will occur.  The goal of the test rig is to create a foil pattern for desired motion of the unmanned underwater vehicle.  Various phase angles will be tested to produce the desired angular movement of the UUV for low speed maneuverability.  Additionally, the results from experimentation will result in the compilation of a blending algorithm for seamless operation at low speed.  The algorithm will incorporate horizontal and vertical movement to ensure 3-axis functionality.  The ultimate goal is for the blending algorithm and the phase angle pattern is to enable low speed maneuverability of the UUV.  Low speed maneuverability enables recovery of UUVs and improved functionality in confined environments.  

The test rig will have a tethered power source.  This power source will also eliminate the need to hardcode the mbed or raspberry pi.  A tether power source will simplify data collection and streamline the experimentation process in order to complete more testing to obtain a larger pool of results.  Waterproof servos will attach to the UUV body.  Communicating using the raspberry pi or mbed, the servo will move the fish fin to the desired phase angle and the ensuing motion will be recorded via accelerometers and video footage.  The video will be analyzed in Matlab via the DLTviewer application.  To simplify testing, the rig will focus on a single plane.  Once mastery is achieved over a single plane, another plane will become the focus.  As experimentation of each plane is completed, results will be blended to create the algoithm motion.  Completed results will be analyzed to to identify the foil pattern for desired motion of the UUV. 





\section{Results}

\subsection{Simplified 2D model}
An example control map from the 2D analysis in \Matlab\ is shown in \fref{fig:controlmap}. 
\begin{figure}
\begin{center}
\includegraphics[width=\columnwidth]{figures/fig4.eps}
\end{center}
\caption{Example control map from 2D analysis in Matlab. Map will eventually be used for high level motion planning.}
\label{fig:controlmap}
\end{figure}


\subsection{3-axis test rig}
An exploded view of the initial prototype 3-axis rig is shown in \fref{fig:rig}. The rig has two or three axes and a modular, removable fin. For waterproof testing, it is actuated with Hitec HS646WP standard servo actuators (waterproof) controlled by an mbed, Arduino Uno, or Raspberry Pi Zero implementing ROS2 and rosserial. 
\begin{figure}
\begin{center}
\includegraphics[height=\columnwidth]{figures/fig6.png}
\end{center}
\caption{Exploded view of initial prototype of bioinspired fin for hydrodynamics testing. First axis provides lateral movement (sway); second axis provides rotation (yaw).}
\label{fig:rig}
\end{figure}
The modular fin connection potentially allows an instrumented smart fin, with proprioception, force \& flow sensing. It also allows connection of alternative fins to test the effect of profile shape, aerofoil section, and fin flexibility and stiffness.

Servo actuators will attach to the body of the UUV.  The actuators are connected to the fins by modular mount pieces (currently Tetrix or Actobotix standard parts).  The fins will passively pop out and adjust its angular position to the desired phase angle sent to the servo.  Servos will receive desired angles from the microcontroller (mbed, Arduino, or Raspberry Pi Zero) to move the position of the fins to the phase angle. This approach to control is leveraged from Evangelista's previous work on Sailbot next generation actuator concepts. 





\subsection{Hydrodynamic modeling of the base UUV}
\Fref{fig:simulink} shows the Simulink model for base UUV hydrodynamic simulation. Its simplifies to a net force in the $x$-direction and net force in the $y$-driection.  Then from $f=ma$ to get acceleration and integrated for velocity.  Advance ratio follows the advance ratio formula.  Lift and drag follow typical lift and drag formulas.
\begin{figure}
\begin{center}
\includegraphics[width=\columnwidth]{\myroot/figures/BasicUUVsimulink.png}
\end{center}
\caption{Simulink diagram for basic UUV simulation.  Model complicates as drag and lift forces change as a result of foil angle.  There is uncertainty about calculation and insertion of thrust forces, so it is included as a gain block set to 2x the drag.}
\label{fig:simulink}
\end{figure}


\subsection{Modular fin designs}
Current modular fin design examples are shown in \fref{fig:findesigns}. We envision making smart fins combining rigid and flexible materials, with embedded sensors, and designed to modularly mount on the actuators at the base. 
\begin{figure}
\begin{center}
\includegraphics[width=\columnwidth]{figures/fig7.png}
\end{center}
\caption{Solid model of example fin design and resulting 3D print. Modular design allows swap out of alternative fin shapes, stiffnesses, or smart fins with embedded instrumentation.}
\label{fig:findesigns}
\end{figure}








\section{Discussion}
Throughout the term, our early explorations have provided much insight useful for final project completion in Spring 2021. Major issues for the spring include
\begin{itemize}
\item Actuator testing
\item Testing on free, cylindrical body
\item Use of Hopper Hall SURF
\item Synergy with turtle fin amphibious effort
\item Initial instrumented smart fin.
\end{itemize}

\subsection{Underwater considerations}
Throughout our research and simulation development this semester, we have learned about the complexities surrounding underwater vehicle modeling and experimentation. One of the first considerations for this project was that since we are looking to increase UUV maneuverability at low speeds, we would have to designate what exactly constitutes low speed. The advance ratio is a non-dimensional parameter that relates the speed of a vehicle through a fluid to the size and rotational speed of its propeller. We determined that an advance ratio under 0.35 would constitute low speed. Additionally, we looked into the possibility of bollard condition affecting our experimentation as fins complete the rowing motion.  At bollard condition, the speed of the UUV is approximately zero.  This ensures the fin use for directionality only, allowing the propeller to be the sole source of velocity.  The angle of attack of the fins will equal the angle the torpedo is holding at the bollard condition.  Developing a sound understanding of the bollard condition facilitates knowledge of the force generated from the flapping fins.  The net force for the fin is the power stroke force plus the recovery stroke force.  Because the force from the power stroke exceeds that of the recovery stroke, net force is in the direction of the power stroke.  The presence of lift and drag forces as a function of attack angle complicate the bollard condition and require more research for sound comprehension.  The bollard condition is applied to the toy sim will be useful for 3-axis hardware experimentation. 

We also could not assume neutral buoyancy as we can in modeling. As we develop our test rig, it is likely that the submerged volume and mass will change. We began planning for what would be the easiest way to achieve neutral buoyancy and came up with foam-filling as previously mentioned. 

\subsection{Simulation issues}
Simulating a UUV with fins is a complex process reliant on accurate, representative equations of the forces acting on vehicles underwater. Our project is primarily concerned with the effect of thrust forces on the UUV body created by fins. This requires a process of researching applicable drag coefficients, implementing equations, accounting for forces and moments, and determining which outputs will tell us the fins are adding maneuverability to the UUV. This is an ongoing process with plenty room for additional considerations and development. Expected simulation challenges inlude
\begin{enumerate}
\item thrust forces
\item drag coeffecients and considerations
\item control algorithm connecting fins to directionality
\item Raspberry Pi ROS2 and Gazebo learning curve issues
\end{enumerate}



\subsection{Modular fin design and complete UUV design}
The design of the fins for our project are inspired by the fins of the mentioned \emph{Mola mola} and triggerfish. We have been familiarizing ourselves with 3D printing and creating CAD files. However, an important attribute of fish fins is their ability to introduce flexure. Flexural stiffness in fins can affect the propulsive and agility capabilities of fish. Optimized stiffness typically result from non-uniform fin profiles with stiff leading edges. Achieving this in a 3D printed fin is difficult and has led us to resin printing. We hope to implement flexure in our test fins using its capabilities of adhesion control and increased precision. 

There are no current results from the 3-axis test rig.  The intention is to get into the Hopper Hall SURF early in the spring to begin test rig.  Results for the test rig come in the form of 3D printed fins.  Due to COVID-19 restrictions, limited progress could be made in 3D printing fins.  Additionally, there was a need to relearn using Autodesk and Makerbot.  An elementary fin was printed, but details of dimensions need refined.  Moreover, the plastic material common for 3D printing creates a rigid structure inconsistent with the flexible material of fish fins.  To alleviate this inconsistency, the intention is to utilize resin material to print appropriately dimensioned fish fins mirroring the pectoral fins of the \emph{Mola mola} or triggerfish.

Following testing of the disembodied fin on a 3 axis test rig, we wish to test on a cylindrical midbody high $L/D$ UUV shape. The elementary version of the UUV will be a 4-inch diameter schedule 12 PVC pipe.  The pipe will be capped at the ends.  Utilizing blue foam, the fundamental UUV will be neutrally buoyant in water.  Neutral buoyancy will also need to incorporate the density of the servo motors and the Tetrix pieces used.  3D printed fish fins will attach to the UUV body with a modular mount system.  The mounts will be attached to a servo motor to enable the fins to reach a desired phase angle for experimentation.  For the recreated fish fins, the current intention is to utilize resin printing to allow for flexibility in the fins to better mimic that found in fish.  The fin structure will be similar to that of an ocean sunfish (\emph{Mola mola}) or a triggerfish. 

\subsection{Optimization, control of fin stroke, and alternative control schemes}
If we consider a complete fin stroke as a trajectory through the control map, can we use variational methods to find an ``optimal'' stroke? For purpose of vehicle design, we wish to design a control (actuator movement) that will maximize some objective ($V$ e.g. force production in a desired direction, efficiency during ahead swimming, etc) subject to constraints on the actuator (maximum extent of motion, maximum speed, or actuator torque limits). We can write this as
\begin{equation}
\mathcal{L} = f(x) + \lambda g(x)
\end{equation}
utilize variational methods to obtain optima, potentially including:
\begin{itemize}
\item \textbf{Highest force} production in a desired direction
\item \textbf{Most efficient} in a desired direction
\item \textbf{Least movement} for a required force
\end{itemize}
as honors-worthy research directions once the rig is working. 

We also plan to consider alternative means of bio-inspired control. While the initial test rig and trials will use high gain servo control of actuator position, is it easier to control based on the actuator speed? Speed gives us what we need to compute forces. Speed must be continuous but need not be without cusps. Speeds give us an instant lookup into our table from before. Alternatively, instead of position control, is it easier to control the actuator based on force feedback, or based on stretch of compliant elementsm, or to play out a feedforward set of stiffnesses about an equilibrium point? 


\subsection{Challenges going forward}
Going forward next semester, we want to finish our UUV model and experimentally validate some of our findings through a 3-axis test rig. We want to determine whether our optimized fin designs and actuation holds up in testing. We also want to develop a control algorithm for our fin system that relates fin actuation to a desired direction. If we successfully 
develop this algorithm, we will be able to effect UUV movement in specified directions. This will make the UUV highly maneuverable. Simulation challenges include creating a control algorithm connecting fins to directional commands, accurate thrust calculation and finding appropriate drag coefficients. Experimental challenges include power supply in an underwater environment, effective implementation of fins to produce maneuverability data, COVID-related lab limitations, accurate sensing and adequate data compiling. 

\section*{Acknowledgements}
We thank our advisor D Evangelista and our sponsor W Sandberg. We thank A Laun and D Fredriksson for taking interest in our project and facilitating the simulation creation.  We thank CAPT B Baker for assistance with rapid prototyping. Finally, we thank our sister project to develop turtle flippers, MIDN 1/C C Ashley, G Brothers, and C Douglas; and MIDN 2/C C Smith. The project is also funded by Lockheed Martin.

%\nocite{hello2020world}
\bibliography{IEEEabrv,\myroot/references/honorsfish.bib}
\begin{IEEEbiography}%
[{\includegraphics[width=1in,height=1.25in,clip,keepaspectratio]{\myroot/figures/M213402.jpg}}]%
{James J Kenneally III} is a midshipman at the United States Naval Academy majoring in Robotics and Control Engineering (Honors). Upon graduation, he will begin training as a lieutenant of Marines. 
\end{IEEEbiography}
\begin{IEEEbiography}%
[{\includegraphics[width=1in,height=1.25in,clip,keepaspectratio]{\myroot/figures/M213546.jpg}}]%
{Evan L Klatt} is a midshipman at the United States Naval Academy majoring in Robotics and Control Engineering (Honors). Upon graduation, he will begin training as a naval aviator and a lieutenant of Marines. 
\end{IEEEbiography}
\end{document}
